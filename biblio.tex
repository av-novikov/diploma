\begin{thebibliography}{99}
\addcontentsline{toc}{section}{Список использованных источников}

% Численные методы
\bibitem{petrov_book}Петров И.Б., Лобанов А.И. Лекции по вычислительной математике: Учебное пособие -- М.: Интернет-Университет Информационных Технологий; БИНОМ.Лаборатория знаний, 2013.-523 с.: ил., табл. -- (Серия <<Основы информационных технологий>>)
\bibitem{magomedov}Магомедов К.М., Холодов А.С. Сеточно-характеристические численные методы. -- М.: Наука, 1988, 288 с.
\bibitem{jp}Сиратори М., Миёси Т., Мацусита Х. Вычислительная механика разрушения. // М.: Мир, 1986. -- 334 с.
\bibitem{kukudzhanov_main}Кукуджанов В.Н. Вычислительная механика сплошных сред. -- М.: Издательство физико-математической литематуры, 2008. -- 320 с.
\bibitem{uilkins}Уилкинс М.Л. Расчёт упругопластических течений. В кн.: Вычислительные методы в гидродинамике. -- М.: Мир, 1967. С.212-163.
\bibitem{belocerkovsky}Белоцерковский О.М. Численное моделирование в механике сплошных сред. — М.: Физико-математическая литература. 1994, 442 с.
\bibitem{fedorenko}Федоренко Р.П. Введение в вычислительную физику. М.:Изд-во Моск. физ. -техн. ин-та, 1994, 528 с.
\bibitem{kukudzhanov1}Кукуджанов В.Н. Распространение упругопластических волн в стержне с учётом влияния скорости деформации. -- М.: ВЦ АН СССР, 1967. С.48.
\bibitem{kukudzhanov2}Кукуджанов В.Н. Численное решение неодномерных задач распространения волн напряжений в твёрдых телах//Сообщение по прикладной математике ВЦ. 1976. Вып. 6. С.67.
\bibitem{bahvalov}Бахвалов Н.С., Панасенко Г.П. Осреднение процессов в периодических средах — математические задачи механики композиционных материалов. 1984

% Физика 
\bibitem{novatsky}Новацкий В. К. Теория упругости. — М. : Мир, 1975, c. 105-107.
\bibitem{sedov}Седов Л. И. Механика сплошной среды. Том 1. — М. : Наука, 1970, с. 143.
\bibitem{rabotnov}Работнов Ю.Н. Механика деформируемого твёрдого тела. — М.: Наука, 1988. — 712 с.
\bibitem{landau_lifshits}Ландау Л.Д., Лифшиц Е.М. Теория упругости. М.: Наука, Главная редакция физико-математической литературы, 1965.
\bibitem{ilyushin}Ильюшин А.А. Пластичность. М.: ОГИЗ, Государственное издательство технико-теоретической литературы, 1948
\bibitem{lehnitsky}Лехницкий С.Г. Теория упругости анизотропного тела. Изд. 2-е, Главная редакция физико-математической литературы издательства <<Наука>>, М., 1977, 416 стр.

% Композиты vs. Анизотропия
\bibitem{simamura}Под. ред. С.Симамуры. Углеродные волокна: Пер. с япон. -- М.:Мир, 1987 - 304 с., ил.
\bibitem{pobedrya}Победря Б.Е. Механика композиционных материалов. -- М.:Изд-во Моск. ун-та, 1984. -- 336 с.
\bibitem{bazhenov}Баженов С.Л., Берлин А.А., Кульков А.А., Ошмян В.Г. Полимерные композиционные материалы. - Долгопрудный: Издательский дом Интеллект, 2010, 352 с.
\bibitem{resler}Реслер И., Хардерс Х., Бекер М. Механическое поведение конструкционных материалов. Пер. с нем. Учебное пособие -- Долгопрудный Издательский Дом <<Интеллект>>, 2011. -- 504 с.
\bibitem{serge}Serge Abrate. Impact on composite structures. Cambridge University Press, 2005. 

% Математика
\bibitem{rozhdestvenskiy}Рождественский Б.Л., Яненко Н.Н. Системы квазилинейных уравнений и их приложения к газовой динамике. -- М.: Наука, 1968.--592 с.
\bibitem{kulikovskiy}Куликовский A.Г., Погорелов Н.В., Семенов А.Ю. Математические вопросы численного решения гиперболических систем уравнений.--М.:ФИЗМАТЛИТ, 2001.--608 с.
\bibitem{ogurtsov}Огурцов К.И., Пахоменко Л.С. Анализ упругих волн, возбуждаемых сосредоточенными источниками в анизотропных средах.Распространение упругих и упруго-пластических волн (Материалы 3-го Всесоюзн. симпозиума. Сборник статей.) Т., <<Фан>>, 1969. 448 с.

% Работы Кафедры Информатики
\bibitem{favorskaya}Фаворская А. В. Постановка задачи численного моделирования динамических про-цессов в сплошной линейно-упругой среде с анизотропией сеточно-характеристическим методом // Труды 54-й научной конференции МФТИ: Пробле-мы фундаментальных и прикладных наук в современном информационном общест-ве. — 2011. — Т. 2. — С. 55 – 56.
\bibitem{g}Иванов В.Д., Кондауров В.Н., Холодов А.С. Расчет динамического деформирования и разрушения упругопластических тел сеточно-характеристическими методами. // Математическое моделирование, 2003, Т. 15, № 10.
\bibitem{petrov_tormasov_holodov}Петров  И.Б., Тормасов А.Г., Холодов А.С. О численном изучении нестационарных процессов в деформируемых средах многослойной структуры // Механика твердого тела – 1989, N 4, с. 89-95.
\bibitem{ivanov_kondaurov_petrov_holodov} Иванов В.Д., Кондауров В.И., Петров И.Б., Холодов А.С. Расчет динамического деформирования и разрушения упругопластических тел сеточно-характеристическими методами – Матем. Моделирование № 2:11, 1990, С. 10 – 29
\bibitem{petrov}Петров И.Б. Волновые и откольные явления в слоистых оболочках конечной толщины // Механика твердого тела – 1986, N 4, с. 118-124.
\bibitem{a4} Васюков А.В., Петров И.Б. О разработке параллельной версии сеточно-характеристического метода для трехмерных уравнений механики деформируемого твердого тела. // Сборник научных трудов <<Модели и методы обработки информации>>. М.: МФТИ, 2009. С. 13--17.
\bibitem{a6} Васюков А.В., Петров И.Б., Черников Д.В. О сеточно"=характеристическом численном методе на неструктурированных сетках для задач механики деформируемого твердого тела в случае трех пространственных переменных. // Сборник научных трудов <<Информационные технологии: модели и методы>>. М.: МФТИ, 2010. С. 52--57.
\bibitem{chelnokov}Челноков Ф.Б. Численное моделирование деформационных процессов в средах со сложной структурой: Дисс. ... канд. физ.-мат. наук – М., 2005

\end{thebibliography}
	
