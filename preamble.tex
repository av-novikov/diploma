\section*{Введение}
\addcontentsline{toc}{section}{Введение}
\setcounter{subsection}{0}
	
	В наши дни каждая отрасль промышленности имеет ряд специфических инженерных задач.
	Особое место среди них занимает подбор и исследование свойств конструкционных материалов, основы будущего продукта.
	За несколько последних десятилетий большое распространение получили полимерные композиционные материалы (ПКМ).
	
	Эти материалы, армируемые высокопрочными волокнами, приобретают ряд совершенно новых (по сравнению с металлами и сплавами) характеристик.
	Высокий модуль упругости и продел прочности в рассчёте на единицу массы, делают их весьма привлекательными для использования в авиационной и спортивной технике, автомобилестроении, судостроении, электронике и медицине.
	
	Сейчас можно увидеть как ПКМ все больше и больше вытесняют традиционные алюминиевые сплавы из авиастроения, позволяя создавать более лёгкие и прочные конструкции.
	Около этом 35-50\% деталей фюзеляжа современных военных и гражданских самолётов делается из ПКМ, армированных углеродными волокнами.
	Так, в аэробусе A-320 европейского гиганта Airbus ПКМ составляют 12,5\% массы самолёта.
	
	Применение композитов в авиации, создаёт потребность моделирования соударения композитной обшивки самолёта с другим телом.
	Эта задача относится к динамическим задачам механики деформируемого твёрдого тела (МДТТ).
	В данной работе исследуется распространение волн упругих деформаций в анизотропных твёрдых телах под действием ударной нагрузки.
	Целью работы является представить результаты некоторых численных расчётов картин деформаций слоистых анизотропных тел. 
	Для того чтобы понять особенности моделирования ПКМ необходимо сказать несколько слов об их структуре \cite{simamura}.
	
\subsection*{Структура композита}
	
	Обшивка самолёта обычно включает в себя несколько субпакетов композита, склеенных между собой эпоксидными смолами. 
	Основу каркаса представляют стрингеры, расположенные вдоль всей обшивки на некотором расстоянии друг от друга.
	Отдельно взятый субпакет состоит из одиннадцати монослоёв композита, спресованных в субпакет.
	Каждый мононслой -- это матрица из полимерных смол, армированная углеродными волокнами.
	Волокна укладываются вдоль одного выделенного напрвления, формируя вертикально-трансверсальную анизотропию в монослое.
	Монослои укладываются друг на друга под так, что углы между их выделенными направлениями составляют некоторую, вполне определённую, последовательность значений.
	Таким образом все выделенные направления лежат в плоскости субпакета и отличаются друг от друга поворотом вокруг оси укладки.
	
\subsection*{Численное моделирование}
		
	Волновые процессы, происходящие в таком многослойном материале, при соударении могут привести к возникновению областей разрушения, растрескивания материала.
	В виду существенной анизотропии материала, при таком нагружении конструкция может заметно терять прочность даже при отсутствии видимых повреждений.
	Поэтому расчёт деформаций в подобных конструкциях должен учитывыть полную волновую картину распространения, отражения и поглощения деформаций.
	
	В настоящее время наиболее широкое распространение для данного класса задач получил метод конечных элементов (МКЭ) \cite{jp}. 
	Однако МКЭ хорош для расчёта статических задач, и в ряде случаев не может обеспечить необходимые пространственные и временные разрешения с учётом влияния контактных границ в слоистых материалах.
	В этой связи, для решения системы уравнений МДТТ в данной работе применяется сеточно-характерестический численный метод, позволяющий преодолеть указанные трудности \cite{magomedov, petrov_book, petrov_tormasov_holodov, petrov}.
	
\subsection*{Обзор литературы}
	
	Исчерпывающую информацию о физике деформаций, происходящих в твёрдом теле можно вынести из фундаментальных книг по МДТТ \cite{rabotnov,novatsky,sedov}.
	Теория пластичности описана в \cite{ilyushin}.
	Теория упругости для анизотропных тел подробно разобрана в \cite{lehnitsky}.
	Механика деформаций применительно к композиционным материалам описана в \cite{resler,pobedrya,bazhenov,simamura}.
	Там же содержатся основные сведения о механике анизотропных тел.
	
	Математическая модель МДТТ в самом общем виде представляет собой гиперболическую систему квазилинейных уравнений в частных производных.
	Описание квазилинейных систем можно найти в \cite{rozhdestvenskiy}. Математические аспекты численного решения гиперболических систем изложены в \cite{kulikovskiy}.
	
	Подробное и последовательное изложении теории численных методов и её приложений есть в \cite{petrov_book}.
	Решению систем гиперболических уравнений сеточно-характеристическим методом посвящена книга \cite{magomedov}.
	Численное моделирование механики сплошных сред излагается в \cite{belocerkovsky,kukudzhanov_main}.
	Стоит отдельно отметить книгу \cite{kukudzhanov_main}, в которой лаконично описаны как и физика сплошных сред, так и численные методы, использемые в механике сплошных сред.
	Основной акцент автор сделал на моделировании больших деформаций и разрушения материалов, что оказалось безмерно полезно при создании этой работы.
	В работе \cite{petrov_tormasov_holodov} исследуется прохождение импульса сжатия через различные многослойные конструкции в одномерной и двумерной постановке для упругого и упругоидеальнопластического материала. Описаны эффекты кумуляции вторичных волн и сепарации.
	В \cite{ivanov_kondaurov_petrov_holodov} сформулирована математическая модель упруговязкопластической повреждающейся среды, учитывающая конечные деформации материала. 
	Для рассчётов испульзуется гибридная схема.
	Такие схемы, позвяющие считать разрывные решения с первым порядком точности, а гладкие со вторым, описаны в \cite{fedorenko}.
	Статья \cite{kukudzhanov1} посвящена изучению распространения одномерных упругопластических волн с учётом влияния скорости деформации. 
	Скорость деформаций существенно влияет на уравнения состояния твёрдых тел, особенно металлов и различных полимеров.
	Полноценная модель упруго-вязко-пластической среды в многомерном случае описана в \cite{kukudzhanov2}. 
	Для построения решения используются характеристические поверхности. Доказана устойчивость, рассмотрены преимущества в сравнении с упруго-пластической моделью.
	
	
	Основными работы, посвящёнными разработке численных методов и пакетов программ моделирования МДТТ в случае трёх пространственных переменных с явным выделением контакта, послужвишими фундаментом создания этой, являются \cite{a4,a6}.
	В \cite{a6} описывается численное решение уравнений МДТТ в трёхмерной постановке сеточно-характеристическим методом на тетраэдрических неструктурированных сетках.
	В \cite{a4} обсуждаются технические аспекты разработки параллельной версии сеточно-характерестического метода. 
	Статья \cite{favorskaya} посвящена численному моделированию линейно упругого анизотропного тела с применением сеточно-характеристического метода.
	
