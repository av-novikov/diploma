\section*{Глава 1\\Основные уравнения механики деформируемого твёрдого тела}
\addcontentsline{toc}{section}{Глава 1. Основные уравнения механики деформируемого твёрдого тела}
\setcounter{section}{1}
\setcounter{subsection}{0}
\setcounter{equation}{0}

\subsection{Уравнения механики деформируемого твёрдого тела}
	
	Основные определяющие соотношения МДТТ включают в себя уравнения движения и реологические соотношения \cite{novatsky,sedov,rabotnov}.
	Посление определяют сопротивление материала по отношению к внешнему воздействию.
\begin{align}
	\label{initial_equations}
	\rho\dot{v}_i &= \nabla_j\sigma_{ij}+f_i & \textrm{(уравнения движения)}\nonumber\\
	\dot{\sigma}_{ij} &= q_{ijkl}\dot{\varepsilon}_{kl}+F_{ij} & \textrm{(реологические соотношения).}
\end{align}
	Здесь $\rho$ – плотность среды, $v_i$ – компоненты скорости смещения,
	$\sigma_{ij}$, $\varepsilon_{ij}$ -- компоненты тензоров напряжений и деформаций,
	$\nabla_j$ – ковариантная производная по $j$-й координате, $f_i$ – массовые
	силы, действующие на единицу объёма, $F_{ij}$ -- правая часть, используемая, например, для описания диссипации в моделях с учётом вязкости.

	В случае малых деформаций тензор скоростей деформаций $e_{ij}=\dot{\varepsilon}_{ij}$ 
	выражается через компоненты скорости смещения линейным образом \cite{landau_lifshits}:
	
\begin{equation}
	\label{small_deformation}
	e_{ij}=\frac{1}{2}(\nabla_j v_i+\nabla_i v_j).
\end{equation}

	Вид компонент тензора 4-го порядка $q_{ijkl}$ и правой части $F_{ij}$ определяется реологией среды.

	Для замыкания системы уравнений \eqref{initial_equations} её необходимо дополнить уравнением состояния, определяющим зависимость плотности от напряжений:

\begin{equation}
	\rho=\rho_0e^{\frac{p}{K}},
\end{equation}
	где $p=-\frac{1}{3}\sum\sigma_{kk}$ -- давление, $K=\lambda+\frac{2}{3}\mu$ -- коэффициент всестороннего сжатия, $\lambda$ и $\mu$ -- параметры Ламе.

	Параметры Ламе зависят от материала и связаны с модулем продольной упругости и коэффициентом Пуассона следующим образом:
\begin{align}
	\label{lame_parameters}
	\lambda &= \frac{E\nu}{(1+\nu)(1-2\nu)}
	\nonumber\\
	\mu &= G=\frac{E}{2(1+\nu)}.
\end{align}
	Здесь $E$ -- модуль продольной упругости, $\nu$ -- коэффициент Пуассона, $G$ -- модуль сдвига.
\clearpage
\newpage

\subsection{Приближение линейно упругого тела}
	
	В простейшем случае линейной упругости тензор $q_{ijkl}$ и правая часть $F_{ij}$ в \eqref{initial_equations} принимают следующий вид \cite{landau_lifshits}:
\begin{align}
	\label{tensor_qijkl_elastic}
	q_{ijkl}&=\lambda\delta_{ij}\delta_{kl}+\mu(\delta_{ik}\delta_{jl}+\delta_{il}
	\delta_{jk}),\nonumber\\
	F_{ij}&=0.
\end{align}
	В этом соотношении $\lambda$ и $\mu$ -- параметры Ламе, $\delta_{ij}$ -- символ Кронекера.
	
	Уравнения \eqref{initial_equations} в приближении \eqref{small_deformation} будут выглядеть следующим образом: 
\begin{align}
	\label{simple_equations}
	\frac{\partial{v_x}}{\partial{t}}&=\frac{1}{\rho}(\frac{\partial{\sigma_{xx}}}{\partial{x}}+\frac{\partial{\sigma_{xy}}}{\partial{y}}+\frac{\partial{\sigma_{xz}}}{\partial{z}})
	\nonumber\\
	\frac{\partial{v_y}}{\partial{t}}&=\frac{1}{\rho}(\frac{\partial{\sigma_{xy}}}{\partial{x}}+\frac{\partial{\sigma_{yy}}}{\partial{y}}+\frac{\partial{\sigma_{yz}}}{\partial{z}})
	\nonumber\\
	\frac{\partial{v_z}}{\partial{t}}&=\frac{1}{\rho}(\frac{\partial{\sigma_{xz}}}{\partial{x}}+\frac{\partial{\sigma_{yz}}}{\partial{y}}+\frac{\partial{\sigma_{zz}}}{\partial{z}})
	\nonumber\\
	\frac{\partial{\sigma_{xx}}}{\partial{t}}&=(\lambda+2\mu)\frac{\partial{v_x}}{\partial{x}}+\lambda\frac{\partial{v_y}}{\partial{y}}+\lambda\frac{\partial{v_z}}{\partial{z}}
	\nonumber\\
	\frac{\partial{\sigma_{xy}}}{\partial{t}}&=\mu(\frac{\partial{v_x}}{\partial{y}}+\frac{\partial{v_y}}{\partial{x}})
	\nonumber\\
	\frac{\partial{\sigma_{xz}}}{\partial{t}}&=\mu(\frac{\partial{v_x}}{\partial{z}}+\frac{\partial{v_z}}{\partial{x}})
	\nonumber\\
	\frac{\partial{\sigma_{yy}}}{\partial{t}}&=\lambda\frac{\partial{v_x}}{\partial{x}}+(\lambda+2\mu)\frac{\partial{v_y}}{\partial{y}}+\lambda\frac{\partial{v_z}}{\partial{z}}
	\nonumber\\
	\frac{\partial{\sigma_{yz}}}{\partial{t}}&=\mu(\frac{\partial{v_z}}{\partial{y}}+\frac{\partial{v_y}}{\partial{z}})
	\nonumber\\
	\frac{\partial{\sigma_{zz}}}{\partial{t}}&=\lambda\frac{\partial{v_x}}{\partial{x}}+\lambda\frac{\partial{v_y}}{\partial{y}}+(\lambda+2\mu)\frac{\partial{v_z}}{\partial{z}}
\end{align}

	Обозначив искомый вектор $\vec{u}=\{v_x,v_y,v_z,\sigma_{xx},\sigma_{xy},\sigma_{xz},\sigma_{yy},\sigma_{yz},\sigma_{zz}\}^T$, уравнения \eqref{simple_equations} можно переписать в матричной форме:

\begin{equation}
	\label{simple_matrix_equation}
	\frac{\partial\vec{u}}{\partial{t}}+\mathbf{A}_x\frac{\partial\vec{u}}{\partial{x}}+
	\mathbf{A}_y\frac{\partial\vec{u}}{\partial{y}}+
	\mathbf{A}_z\frac{\partial\vec{u}}{\partial{z}}=0.
\end{equation}

	Для линейно упругого тела матрицы $\mathbf{A}_x$, $\mathbf{A}_y$, $\mathbf{A}_z$ в \eqref{simple_matrix_equation} принимают следующий вид.

\begin{align}
\label{isotropic_mat1}
\mathbf{A}_x =
\left( \begin{array}{cccccccccccc}
0 & 0 & 0 & -\frac 1 \rho & 0 & 0 & 0 & 0 & 0 \\ 
0 & 0 & 0 & 0 & -\frac 1 \rho & 0 & 0 & 0 & 0 \\ 
0 & 0 & 0 & 0 & 0 & -\frac 1 \rho & 0 & 0 & 0 \\ 
-(\lambda+2\mu) & 0 & 0 & 0 & 0 & 0 & 0 & 0 & 0 \\ 
0 & -\mu & 0 & 0 & 0 & 0 & 0 & 0 & 0 \\ 
0 & 0 & -\mu & 0 & 0 & 0 & 0 & 0 & 0 \\ 
-\lambda & 0 & 0 & 0 & 0 & 0 & 0 & 0 & 0 \\ 
0 & 0 & 0 & 0 & 0 & 0 & 0 & 0 & 0 \\ 
-\lambda & 0 & 0 & 0 & 0 & 0 & 0 & 0 & 0  
\end{array} \right),
\end{align} 
\begin{align}
\label{isotropic_mat2}
\mathbf{A}_y =
\left( \begin{array}{cccccccccccc}
0 & 0 & 0 & 0 & -\frac 1 \rho & 0 & 0 & 0 & 0 \\ 
0 & 0 & 0 & 0 & 0 & 0 & -\frac 1 \rho & 0 & 0 \\ 
0 & 0 & 0 & 0 & 0 & 0 & 0 & -\frac 1 \rho & 0 \\ 
0 & -\lambda & 0 & 0 & 0 & 0 & 0 & 0 & 0 \\ 
-\mu & 0 & 0 & 0 & 0 & 0 & 0 & 0 & 0 \\ 
0 & 0 & 0 & 0 & 0 & 0 & 0 & 0 & 0 \\ 
0 & -(\lambda+2\mu) & 0 & 0 & 0 & 0 & 0 & 0 & 0 \\ 
0 & 0 & -\mu & 0 & 0 & 0 & 0 & 0 & 0 \\ 
0 & -\lambda & 0 & 0 & 0 & 0 & 0 & 0 & 0  
\end{array} \right),
\end{align}
\begin{align}
\label{isotropic_mat3}
\mathbf{A}_z =
\left( \begin{array}{cccccccccccc}
0 & 0 & 0 & 0 & 0 & -\frac 1 \rho & 0 & 0 & 0 \\ 
0 & 0 & 0 & 0 & 0 & 0 & 0 & -\frac 1 \rho & 0 \\ 
0 & 0 & 0 & 0 & 0 & 0 & 0 & 0 & -\frac 1 \rho \\ 
0 & 0 & -\lambda & 0 & 0 & 0 & 0 & 0 & 0 \\ 
0 & 0 & 0 & 0 & 0 & 0 & 0 & 0 & 0 \\ 
-\mu & 0 & 0 & 0 & 0 & 0 & 0 & 0 & 0 \\ 
0 & 0 & -\lambda & 0 & 0 & 0 & 0 & 0 & 0 \\ 
0 & -\mu & 0 & 0 & 0 & 0 & 0 & 0 & 0 \\ 
0 & 0 & -(\lambda+2\mu) & 0 & 0 & 0 & 0 & 0 & 0  
\end{array} \right).
\end{align}\\
	Аналогично можно записать более общую систему \eqref{initial_equations} в виде:

\begin{equation}
	\label{matrix_equation}
	\frac{\partial\vec{u}}{\partial{t}}+\mathbf{A}_x\frac{\partial\vec{u}}{\partial{x}}+
	\mathbf{A}_y\frac{\partial\vec{u}}{\partial{y}}+
	\mathbf{A}_z\frac{\partial\vec{u}}{\partial{z}}=\vec{f}.
\end{equation}
	Здесь $\vec{f}$ -- вектор правых частей, размерность которого равна размерности исходной системы, а выражения для компонентов зависят от реологии среды.
	Точный вид матриц $\mathbf{A}_x$, $\mathbf{A}_y$, $\mathbf{A}_z$ зависит от реологии среды.
	
	Система, записанная в общем виде \eqref{matrix_equation}, представляет из себя \textit{систему квазилинейных уравнений} в частных производных\cite{rozhdestvenskiy}, где $\mathbf{A}_x = \mathbf{A}_x(t, x, y, z, \vec{u})$, $\mathbf{A}_y = \mathbf{A}_y(t, x, y, z, \vec{u})$, $\mathbf{A}_z = \mathbf{A}_z(t, x, y, z, \vec{u})$, $\vec{f} = \vec{f}(t, x, y, z, \vec{u})$ в общем случае.
	Она будет линейной в случае: $\mathbf{A}_x = \mathbf{A}_x(t, x, y, z)$, $\mathbf{A}_y = \mathbf{A}_y(t, x, y, z)$, $\mathbf{A}_z = \mathbf{A}_z(t, x, y, z)$.
	
	Для широкого класса задач МДТТ система уравнений \eqref{matrix_equation} имеет \textit{гиперболический} тип. Это значит, что существуют несколько интегралов движения и система может быть решена методом характеристик.
\clearpage
\newpage

\subsection{Линейно-упругое приближение анизотропного тела}

	Математическая модель анизотропной среды для трехмерного случая описывается системой уравнений\cite{favorskaya}:
\begin{small}
\begin{align}
	\label{anisotropic_equations}
	\frac{\partial{v_x}}{\partial{t}}&=\frac{1}{\rho}(\frac{\partial{\sigma_{xx}}}{\partial{x}}+\frac{\partial{\sigma_{xy}}}{\partial{y}}+\frac{\partial{\sigma_{xz}}}{\partial{z}})
	\nonumber\\
	\frac{\partial{v_y}}{\partial{t}}&=\frac{1}{\rho}(\frac{\partial{\sigma_{xy}}}{\partial{x}}+\frac{\partial{\sigma_{yy}}}{\partial{y}}+\frac{\partial{\sigma_{yz}}}{\partial{z}})
	\nonumber\\
	\frac{\partial{v_z}}{\partial{t}}&=\frac{1}{\rho}(\frac{\partial{\sigma_{xz}}}{\partial{x}}+\frac{\partial{\sigma_{yz}}}{\partial{y}}+\frac{\partial{\sigma_{zz}}}{\partial{z}})
	\nonumber\\
	\frac{\partial{\sigma_{xx}}}{\partial{t}}&=c_{11}\frac{\partial{v_x}}{\partial{x}}+c_{12}\frac{\partial{v_y}}{\partial{y}}+c_{13}\frac{\partial{v_z}}{\partial{z}}+c_{14}(\frac{\partial{v_z}}{\partial{y}}+\frac{\partial{v_y}}{\partial{z}})+c_{15}(\frac{\partial{v_z}}{\partial{x}}+\frac{\partial{v_x}}{\partial{z}})+c_{16}(\frac{\partial{v_y}}{\partial{x}}+\frac{\partial{v_x}}{\partial{y}})
	\nonumber\\
	\frac{\partial{\sigma_{yy}}}{\partial{t}}&=c_{12}\frac{\partial{v_x}}{\partial{x}}+c_{22}\frac{\partial{v_y}}{\partial{y}}+c_{23}\frac{\partial{v_z}}{\partial{z}}+c_{24}(\frac{\partial{v_z}}{\partial{y}}+\frac{\partial{v_y}}{\partial{z}})+c_{25}(\frac{\partial{v_z}}{\partial{x}}+\frac{\partial{v_x}}{\partial{z}})+c_{26}(\frac{\partial{v_y}}{\partial{x}}+\frac{\partial{v_x}}{\partial{y}})
	\nonumber\\
	\frac{\partial{\sigma_{zz}}}{\partial{t}}&=c_{13}\frac{\partial{v_x}}{\partial{x}}+c_{23}\frac{\partial{v_y}}{\partial{y}}+c_{33}\frac{\partial{v_z}}{\partial{z}}+c_{34}(\frac{\partial{v_z}}{\partial{y}}+\frac{\partial{v_y}}{\partial{z}})+c_{35}(\frac{\partial{v_z}}{\partial{x}}+\frac{\partial{v_x}}{\partial{z}})+c_{36}(\frac{\partial{v_y}}{\partial{x}}+\frac{\partial{v_x}}{\partial{y}})
	\nonumber\\
	\frac{\partial{\sigma_{yz}}}{\partial{t}}&=c_{14}\frac{\partial{v_x}}{\partial{x}}+c_{24}\frac{\partial{v_y}}{\partial{y}}+c_{34}\frac{\partial{v_z}}{\partial{z}}+c_{44}(\frac{\partial{v_z}}{\partial{y}}+\frac{\partial{v_y}}{\partial{z}})+c_{45}(\frac{\partial{v_z}}{\partial{x}}+\frac{\partial{v_x}}{\partial{z}})+c_{46}(\frac{\partial{v_y}}{\partial{x}}+\frac{\partial{v_x}}{\partial{y}})
	\nonumber\\
	\frac{\partial{\sigma_{xz}}}{\partial{t}}&=c_{15}\frac{\partial{v_x}}{\partial{x}}+c_{25}\frac{\partial{v_y}}{\partial{y}}+c_{35}\frac{\partial{v_z}}{\partial{z}}+c_{45}(\frac{\partial{v_z}}{\partial{y}}+\frac{\partial{v_y}}{\partial{z}})+c_{55}(\frac{\partial{v_z}}{\partial{x}}+\frac{\partial{v_x}}{\partial{z}})+c_{56}(\frac{\partial{v_y}}{\partial{x}}+\frac{\partial{v_x}}{\partial{y}})
	\nonumber\\
	\frac{\partial{\sigma_{xy}}}{\partial{t}}&=c_{16}\frac{\partial{v_x}}{\partial{x}}+c_{26}\frac{\partial{v_y}}{\partial{y}}+c_{36}\frac{\partial{v_z}}{\partial{z}}+c_{46}(\frac{\partial{v_z}}{\partial{y}}+\frac{\partial{v_y}}{\partial{z}})+c_{56}(\frac{\partial{v_z}}{\partial{x}}+\frac{\partial{v_x}}{\partial{z}})+c_{66}(\frac{\partial{v_y}}{\partial{x}}+\frac{\partial{v_x}}{\partial{y}})
\end{align}
\end{small}
	
	Коээфициенты в последних шести уравнениях - составляющие \textit{тензора упругих постоянных} четвёртого ранга, который в данном случае записывается в виде матрицы:

\begin{align}
\label{anisotropic_tensor}	
c_{ik} =
\left( \begin{array}{cccccccccccc}
c_{11} & c_{12} & c_{13} & c_{14} & c_{15} & c_{16} \\ 
c_{12} & c_{22} & c_{23} & c_{24} & c_{25} & c_{26} \\ 
c_{13} & c_{23} & c_{33} & c_{34} & c_{35} & c_{36} \\ 
c_{14} & c_{24} & c_{34} & c_{44} & c_{45} & c_{46} \\ 
c_{15} & c_{25} & c_{35} & c_{45} & c_{55} & c_{56} \\ 
c_{16} & c_{26} & c_{36} & c_{46} & c_{56} & c_{66}
\end{array} \right).
\end{align}\\
	
	Собрав производные вдоль различных осей по матрицам $\mathbf{A}_x$, $\mathbf{A}_y$, $\mathbf{A}_z$ и снова обозначив искомый вектор $\vec{u}=\{v_x,v_y,v_z,\sigma_{xx},\sigma_{xy},\sigma_{xz},\sigma_{yy},\sigma_{yz},\sigma_{zz}\}^T$, уравнения \eqref{anisotropic_equations} предстанут в виде \eqref{simple_matrix_equation} с соответствующими матрицами:
	
\begin{align}
\label{anisotropic_mat1}	
\mathbf{A}_x = - 
\left( \begin{array}{cccccccccccc}
0 & 0 & 0 & \frac 1 \rho & 0 & 0 & 0 & 0 & 0 \\ 
0 & 0 & 0 & 0 & \frac 1 \rho & 0 & 0 & 0 & 0 \\ 
0 & 0 & 0 & 0 & 0 & \frac 1 \rho & 0 & 0 & 0 \\ 
c_{11} & c_{16} & c_{15} & 0 & 0 & 0 & 0 & 0 & 0 \\ 
c_{16} & c_{66} & c_{56} & 0 & 0 & 0 & 0 & 0 & 0 \\
c_{15} & c_{56} & c_{55} & 0 & 0 & 0 & 0 & 0 & 0 \\ 
c_{12} & c_{26} & c_{25} & 0 & 0 & 0 & 0 & 0 & 0 \\ 
c_{14} & c_{46} & c_{45} & 0 & 0 & 0 & 0 & 0 & 0 \\ 
c_{13} & c_{36} & c_{35} & 0 & 0 & 0 & 0 & 0 & 0
\end{array} \right),
\end{align} 
\begin{align}
\label{anisotropic_mat2}
\mathbf{A}_y = - 
\left( \begin{array}{cccccccccccc}
0 & 0 & 0 & 0 & \frac 1 \rho & 0 & 0 & 0 & 0 \\ 
0 & 0 & 0 & 0 & 0 & 0 & \frac 1 \rho & 0 & 0 \\ 
0 & 0 & 0 & 0 & 0 & 0 & 0 & \frac 1 \rho & 0 \\ 
c_{16} & c_{12} & c_{14} & 0 & 0 & 0 & 0 & 0 & 0 \\ 
c_{66} & c_{26} & c_{46} & 0 & 0 & 0 & 0 & 0 & 0 \\
c_{56} & c_{25} & c_{45} & 0 & 0 & 0 & 0 & 0 & 0 \\
c_{26} & c_{22} & c_{24} & 0 & 0 & 0 & 0 & 0 & 0 \\ 
c_{46} & c_{24} & c_{44} & 0 & 0 & 0 & 0 & 0 & 0 \\
c_{36} & c_{23} & c_{34} & 0 & 0 & 0 & 0 & 0 & 0   
\end{array} \right),
\end{align}
\begin{align}
\label{anisotropic_mat3}
\mathbf{A}_z = - 
\left( \begin{array}{cccccccccccc}
0 & 0 & 0 & 0 & 0 & \frac 1 \rho & 0 & 0 & 0 \\ 
0 & 0 & 0 & 0 & 0 & 0 & 0 & \frac 1 \rho & 0 \\ 
0 & 0 & 0 & 0 & 0 & 0 & 0 & 0 & \frac 1 \rho \\ 
c_{15} & c_{14} & c_{13} & 0 & 0 & 0 & 0 & 0 & 0 \\ 
c_{56} & c_{46} & c_{36} & 0 & 0 & 0 & 0 & 0 & 0 \\
c_{55} & c_{45} & c_{35} & 0 & 0 & 0 & 0 & 0 & 0 \\ 
c_{25} & c_{24} & c_{23} & 0 & 0 & 0 & 0 & 0 & 0 \\ 
c_{45} & c_{44} & c_{34} & 0 & 0 & 0 & 0 & 0 & 0 \\ 
c_{35} & c_{34} & c_{33} & 0 & 0 & 0 & 0 & 0 & 0  
\end{array} \right).
\end{align}\\

	Подробнее, общий случай анизотропного упругого тела и различные типы анизотропии будут рассмотрены в Главе 2.
